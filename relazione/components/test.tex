\section{Fase di testing}
Particolare attenzione è stata dedicata alla fase di validazione del sito web realizzato: tutte le pagine sono state sottoposte ad un accorta validazione. Per fare ciò, oltre all'occhio umano, sono stati usati i seguenti strumenti: 
\begin{itemize}
	\item per la validazione del codice XHTML è stato usato il validatore di W3C, Markup Validation Service;
	\item per la validazione dei fogli di stile in CSS è stato usato il validatore di W3C, CSS Validation Service;
	\item per il controllo dei livelli di contrasto di colore presenti nel sito sono state usate l'estensione Wave - web accessibility evaluation tool e l'estensione del browser Mozilla Firefox Totally Automated A11y Scanner;
	\item per la simulazione di alcune disabilità visive è stata usata l'estensione Silktide - disability simulator.
\end{itemize}
Inoltre è stata testata la compatibilità del sito sui browser più utilizzati. Si garantisce, quindi, il funzionamento del sito per i seguenti browser:
\begin{itemize}
\item Chrome;
\item Mozilla Firefox;
\item Edge.
\end{itemize}
\subsection{NOTA}Su Mozilla Firefox, nell'anteprima di stampa, ad un livello di zoom normale, non vengono renderizzati correttamente i bordi delle tabelle. Questo è causato dal visualizzatore di Firefox, in quanto si può notare, con un zoom superiore, che i bordi sono corretti.  